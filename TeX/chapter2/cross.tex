% !TEX encoding = UTF-8 Unicode 
% !TEX root = praca.tex

\subsection{Cross-platform mobile development}

As the term suggests, cross-platform or multi-platform frameworks enable to create applications that can be installed on different platforms, and consequently reach a larger user base. There are many solutions available in the market, which all offer support for a narrow or wide subset of existing platforms. Therefore, applications may be published as, e.g., mobile apps (Android, iOS), web apps, desktop apps (Windows, macOS, Linux) or even embedded apps. The number of operating systems tends to increase under the principle of ``write once, run everywhere'' followed by frameworks' creators \cite{comparison_technologies_multiplatform}. 

The main advantage of cross-platform development over native development is in line with its primary goal, which is the ability to create and maintain a single codebase, no matter the number of target platforms. Moreover, as described in the chapter \ref{chap:native}, Android operating system suffers from high level of fragmentation which can be addressed with cross-platform framework as well. Being able to operate on a single codebase results in the development costs decrease. Implementation is more time-efficient without the need to build multiple projects. This remains true for post-release support in the form of updating versions and handling bugs or change requests. Cross-platform approach is also lighter on resources, requiring just a single development team, which additionally removes possible collaboration issues between teams that can occur in native approach. In the past, it would be assumed it is easier to gather an experienced team for native development, however currently the most popular multi-platform frameworks are mature enough to have created active communities with high level of know-how \cite{approach_to_assess_performance_case_study,comparison_perf_looks_flutter_native,comp_analysis_hybrid_frameworks}.

% TODO: disadvantages: native may provide better connection with native components and services mostly depending on type of cross-platform framework which may be the cause of poorer performance

% TODO: Different types 

There are differences between available cross-platform frameworks, especially in the context of 
architecture or compilation method. They will be described in the following chapters. Essentially, most of them require a middle layer that connects the app with the system and translates the commands to be natively called. This is considered to be a possible root of performance decrease \cite{appdynamics_mobile_app_performance}. And although a single workspace is used during development and testing, in order to publish the final application to different app stores (e.g. Google Play Store, App Store), there must be performed a build for each target platform to acquire separate app bundles that can then be uploaded \cite{comp_study_hybrid}.

As explained in the previous chapter, different operating systems usually have different guidelines provided when it comes to user interface (UI) design. This might become an issue depending on design assumptions. Considering the user's point of view, there are three approaches to UI implementation. Firstly, the app appearance may be identical regardless of the platform it runs on. In this case, the platform-specific conventions may not be fulfilled and only when implemented correctly it will not cause user experience decrease. Secondly, the app may have a completely system-compliant look and feel. In this case, the issue may be raised as to how efficient it is to implement disjointed layouts in the cross-platform solution compared to switching to the native approach. Finally, there is an intermediate approach which assumes that most of the app elements are shared between platforms, but there are some that are platform-specific, e.g., popups, action buttons \cite{cross_platform_ux,baseline_cross_platform}.



\subsubsection{Flutter}



\subsubsection{React Native}



\subsubsection{Ionic}



\subsubsection{Comparison}

\begin{table}[hb]
	\centering
    \caption{Cross-platform frameworks comparison (Source: Own work based on \dots)}
    \label{tab:framework_comparison}
	\begin{tabular}{ |l|p{30mm}|p{30mm}|p{30mm}| }
		\hline
        \diagbox{Element}{Framework} & Flutter & React Native & Ionic \\
		\hline
		Initial release&2017&&\\
        \hline
		Current stable version&&&\\
        \hline
		Implemented with&C, C++, Dart&&\\
        \hline
		Supported platforms&Android, iOS, Web, Windows, macOS, Linux&&\\
        \hline
		Supported IDEs??&&&\\
        \hline
		Programming language&Dart&&\\
        \hline
		Rendering&Canvas drawing&Native platform components&Native platform components\\
		\hline
	\end{tabular}
\end{table}
