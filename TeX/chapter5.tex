% !TEX encoding = UTF-8 Unicode 
% !TEX root = praca.tex

\chapter{Research method}

Literature review concluded in Chapter \ref{chap:eval_cross_platform} portrays the process of cross-platform framework evaluation as a multifaceted problem. The reason being, there are numerous aspects (criteria) that must be considered. Therefore, different major research approaches may be distinguished. A potential approach could revolve around choosing a single cross-platform framework that would get extensively analyzed in accordance with those criteria. Another method could be to select a few frameworks for a comparison while restricting the criteria to a concise subset that will decrease the complexity and labor consumption but still represent a relevant research perspective. The latter approach is the one applied for the purpose of this thesis. While various criteria from all the perspectives listed in Table \ref{tab:eval_criteria} are accounted for within the framework descriptions in Chapter \ref{chap:cross_mobile_dev}, this thesis's research focuses on the \emph{Performance} criterion. Some other criteria answered indirectly are: supported platforms, user interface design, access to platform-specific functionality, as well as look and feel.

\section{Performance metrics}\label{chap:performance_metrics}

In this chapter, specific performance metrics are chosen according to the literature review conducted to provide a knowledge base for Chapter \ref{chap:app_performance}.

\bigskip

The following metrics are proposed for the research on mobile application performance:
% TODO: app size??
\begin{itemize}
    \item CPU usage
    \item Memory consumption
    \item Power consumption
    \item Frame rate (FPS)
\end{itemize}

The variety of those metrics should allow for diversified experiments with valuable outcomes regarding multiple mobile application performance aspects.

\section{Research scenarios}\label{chap:research_scenarios}

Naturally, the best solution to acquire real-life performance data per each tested technology would be to use multiple complete real-life applications. However, implementation of even one of such application would require a lot of work and arguably a whole team of specialists responsible for different aspects of the development. In order to somehow abstract the testing environment, the most common and significant functionalities can be extracted from available mobile applications to provide a list of elements that should be tested.

\bigskip

The following research scenarios are proposed in relation to the selected functionalities:

\begin{itemize}
    \item \textbf{Research scenario 1: List scrolling and filtering}
        
    One of the most common elements of mobile applications is a scrollable list view. It can be found in social media apps, news apps, games and e-commerce apps. Additionally, quite often a mechanism of filtering those lists by the user is applied. For this research scenario, a view should be constructed with a scrollable and filterable list of randomly generated text data.

    \bigskip

    \item \textbf{Research scenario 2: Animations}
    
    Currently, a big emphasis is put on the look and feel of mobile applications in order to guarantee the highest level of user experience. One of the crucial aspects are the animations, which both improve the visuals but also provide the user with the feedback resulting from the actions performed. For this research scenario, a view should be constructed with exemplary animations utilized, e.g., element size change, rotations.

    \bigskip

    \item \textbf{Research scenario 3: File I/O (Input \& Output)}
    
    Another important functionality that can be found in many popular applications is the ability to upload files to the app from the device storage, as well as save files with some app data on the device. For this research scenario, a view should be constructed with the buttons enabling uploading and presenting an image from the device and saving a file with some text data.

    \bigskip

    \item \textbf{Research scenario 4: Common UI elements}
    
    User interface layouts are built with many different components, such as toggle buttons, drop-down buttons, and text fields. One of the most common usage of text fields are fillable forms that can be found in registration process, for example. Moreover, almost always there are multiple pages that can be navigated between. For this research scenario, a view should be constructed with a bottom navigation bar enabling switching between two views. The first view should consist of various controls, e.g., toggle buttons, and the second view should include a text field form. Additionally, there should be a button enabling navigating to a separate page.

\end{itemize}

\section{Testing tool}

IDEs such as Android Studio and Xcode offer built-in profilers that can be utilized for performance testing during mobile application development. However, in case of analyzing more than one platform, multiple profilers would have to be used as those IDEs support only a single operation system each. Therefore, in order to produce the most comparable results, \emph{GameBench Studio PRO}, a standalone testing tool, has been selected for the purpose of this thesis.

\emph{GameBench Studio PRO} is a performance monitoring tool which can be used for both Android and iOS applications. It is mostly dedicated to mobile game profiling but nevertheless it can be successfully used with virtually any mobile app. Some of the key features offered by the tool are: a Web dashboard for session management and a wide range of supported metrics (CPU, GPU, RAM, Battery and Network usage, Frame rendering, Data download and upload).

\section{Testing devices}

Table \ref{tab:testing_devices} contains technical specification of the devices used as the environment for experiments.

\begin{longtblr}[
    caption = {Testing devices (Source: Own work based on \cite{mgsm_apple_iphone_7,mgsm_apple_iphone_13_mini,mgsm_sony_xperia_z1,mgsm_xiaomi_9t_pro})},
    label = {tab:testing_devices},
]{ colspec = { |l|X|X|X|X| }, hlines}
    \textbf{Device}&Sony Xperia Z1&Xiaomi Mi 9T Pro&iPhone 7&iPhone 13 mini\\
    \textbf{Released}&Q3 2013&Q3 2019&Q3 2016&Q3 2021\\
    \textbf{OS}&Android 5.1.1&Android 11&iOS 15.7.3&iOS 16.5\\
    \textbf{CPU}&Qualcomm Snapdragon 800 (2.20GHz, 4-core)&Qualcomm Snapdragon 855 (2.84GHz, 8-core)&Apple A10 Fusion (2.37GHz, 4-core)&Apple A15 Bionic (3.22GHz, 6-core)\\
    \textbf{RAM}&2GB&6GB&2GB&4GB\\
    \textbf{Display}&TFT 1080x1920px (5.00") 441ppi&AMOLED 1080x2340px (6.39") 403ppi&IPS TFT 750x1334px (4.70") 326ppi&OLED 1080x2340px (5.40") 477ppi\\
\end{longtblr}

The devices were selected in a way to cover different price points and ages so that the research can lead to the most general results possible, having regard to the restrictions caused by the limited access to devices.

\clearpage
