% !TEX encoding = UTF-8 Unicode 
% !TEX root = praca.tex

\chapter{App performance measurement}

\section{Mobile development}

The possibility to install and uninstall mobile apps within seconds directly affects the commercial mobile app market. In order to compete, publishers need to make sure they provide their services at the highest quality acquirable so that users don't turn to the competitor's solution. Almost 30\% of consumers instantly switch to other available products if their needs are not satisfied. App performance is considered to be one of the more important aspects in this context as 70\% of users perform an immediate switch solely based on loading time being too long. For that reason, no matter how big or successful, each mobile app publisher must not underestimate the importance of performance offered by their product \cite{micro_moments_guide}. 

Mobile app performance is a broad term and can be understood differently and described with higher or lower level of detail. It can be seen as only dependent on consumers' impressions. The main aspects of app performance as perceived by users are responsiveness, loading times, device resource usage, smoothness, crash occurrence, <different metrics>?

In multiple literature positions, statements can be found about mobile app performance being significantly higher when developing using native approach compared to cross-platform approach. However, even in many rather recent papers this statement is assumed based on the work of P. Que, X. Guo and M. Zhu (\cite{que_comp_hybrid_native}) which was published in 2016. Across the past 7 years there have been many breaking changes in both native and multi-platform solutions. It is reasonable to assume they may have changed that situation therefore it is worthwhile to review the current state-of-art and draw present-day conclusions.

\section{Web development}
