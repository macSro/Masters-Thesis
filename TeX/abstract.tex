\abstract{

There are various aspects affecting the overall perception of quality of a mobile application, with performance being one of the most significant, especially from the perspective of the user. Having that in mind, it is crucial to understand the differences between the available mobile development approaches and in which use cases they are able to provide the highest value.

The purpose of this master's thesis was to perform a comparative analysis of the performance of mobile applications built using both native and cross-platform solutions. Exemplary applications were implemented with Kotlin, Swift, Flutter, React Native, and Ionic to be used as the environment for the experiments. The experiments provided results considering the selected performance metrics, e.g., CPU, memory, and power usage. The results were interpreted in order to find benefits and/or weaknesses for each studied solution, as well as to try to define optimal scenarios for their use.

}{

Na ogólne postrzeganie jakości aplikacji mobilnej wpływają różne aspekty, przy czym wydajność jest jednym z najistotniejszych, zwłaszcza z perspektywy użytkownika. Mając to na uwadze, kluczowe jest zrozumienie różnic pomiędzy dostępnymi podejściami do wytwarzania aplikacji mobilnych i w jakich przypadkach użycia są one w stanie zapewnić najwyższą wartość.

Celem niniejszej pracy magisterskiej było przeprowadzenie analizy porównawczej wydajności aplikacji mobilnych zbudowanych z wykorzystaniem rozwiązań natywnych oraz cross-platformowych. Przykładowe aplikacje zostały zaimplementowane przy użyciu Kotlin, Swift, Flutter, React Native oraz Ionic, aby posłużyć jako środowisko do przeprowadzenia eksperymentów. Eksperymenty dostarczyły wyniki uwzględniające wybrane metryki wydajności, np. zużycie procesora, pamięci i energii. Wyniki zostały zinterpretowane w celu znalezienia korzyści i/lub słabości dla każdego badanego rozwiązania, a także próby zdefiniowania optymalnych scenariuszy ich wykorzystania.

}