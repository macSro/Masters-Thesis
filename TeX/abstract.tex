\abstract{

There are various aspects affecting the overall perception of quality of a mobile application, with performance being one of the most significant, especially from the perspective of the user. Having that in mind, it is crucial to understand the differences between the available mobile development approaches and in which use cases they are able to provide the highest value.

The purpose of this master's thesis was to perform a comparative analysis of native and cross-platform approaches to mobile development. The primary basis for the comparison was the performance of applications created using those methods. Exemplary applications were implemented with Kotlin, Swift, Flutter, and React Native, to be used as the environment for the experiments. The experiments provided results considering the selected performance metrics, CPU usage, memory and power consumption, and frame rate stability. The results were interpreted in order to find benefits and/or weaknesses of the studied technologies, as well as to try to define optimal scenarios for their use.

}{

Na ogólne postrzeganie jakości aplikacji mobilnej wpływają różne aspekty, przy czym wydajność jest jednym z najistotniejszych, zwłaszcza z perspektywy użytkownika. Mając to na uwadze, kluczowe jest zrozumienie różnic pomiędzy dostępnymi podejściami do wytwarzania aplikacji mobilnych i w jakich przypadkach użycia są one w stanie zapewnić najwyższą wartość.

Celem niniejszej pracy magisterskiej było przeprowadzenie analizy porównawczej natywnych oraz wieloplatformowych metod wytwarzania aplikacji mobilnych. Jako główną podstawę porównania wybrano wydajność aplikacji, zaimplementowanych przy ich użyciu. Przykładowe aplikacje zostały zbudowane przy użyciu Kotlin, Swift, Flutter oraz React Native, aby posłużyć jako środowisko do przeprowadzenia eksperymentów. Eksperymenty dostarczyły wyniki uwzględniające wybrane metryki wydajności, zużycie procesora, pamięci i energii, oraz stabilność częstotliwości odświeżania. Wyniki zostały zinterpretowane w celu znalezienia korzyści i/lub słabości badanych technologii, a także próby zdefiniowania optymalnych scenariuszy ich wykorzystania.

}