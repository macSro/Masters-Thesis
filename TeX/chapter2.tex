% !TEX encoding = UTF-8 Unicode 
% !TEX root = praca.tex

\chapter{Related work}

\begin{table}[h]
	\centering
    \caption{Related work (Source: Own work)}
    \label{tab:related_work}
	\begin{tabular}{ |c|p{.87\textwidth}| }
		\hline
        \textbf{Paper}&\textbf{Key takeaways}\\
        \hline
        \cite{comparison_perf_looks_flutter_native}&\\
        \hline
        \cite{eval_rn_flutter}&\\
        \hline
        \cite{cross_platform_development_study_rn_flutter}&\\
        \hline
        \cite{comparison_technologies_multiplatform}&\\
        \hline
        \cite{denko_comp_hybrid}&\\
        \hline
        \cite{comp_analysis_hybrid_frameworks}&\\
        \hline
        \cite{bialkowski_eval_flutter}&\\
        \hline
        \cite{kocki_comp_hybrid_ios}&\\
        \hline
        \cite{approach_to_assess_performance_case_study}&D. Mota and R. Martinho introduce a stable approach to performance assessment of mobile apps developed with cross-platform frameworks. They propose a set of metrics (CPU usage, RAM usage, execution time, FPS), as well as evaluation features and result normalization method. Thereafter, the proposed solution is applied to compare Flutter and React Native. \\
        \hline
        \cite{willocx_quantative_perf}&M. Willocx, J. Vossaert and V. Naessens propose an approach to mobile app performance assessment on the example of Xamarin and PhoneGap. Based on the results some guidelines for framework selection are suggested. The performance metrics selected are: launch time, pause and resume time, time to open page, memory consumption, CPU usage and disk space.\\
        \hline
        \cite{hort_survey_perf_optimization}&M. Hort, M. Kechagia, F. Sarro and M. Harman describe a variety of mobile application performance optimization techniques inspired by the literature from 2008-2020. The metrics considered are responsiveness, launch time, memory usage and energy consumption.\\
        \hline
        \cite{survey_taxonomy_cross_platform}&A. Biørn-Hansen and T-M. Grønli and G. Ghinea provide detailed definitions of different approaches to cross-platform development as well as correct misconceptions found in other literature. For a group of concepts (User Experience, Software Platform Features, Performance and Hardware, Security) the state-of-research is described and suggestions for future work are proposed.\\
        \hline
        \cite{rieger_eval_cp}&C. Rieger and T. A. Majchrzak propose a detailed framework consisting of 33 criteria divided into 4 perspectives: Infrastructure, Development, App and Usage. The purpose of such framework is to be used for the complete assessment of a cross-platform solution. In this case completeness means considering both technical and business aspects.\\
		\hline
        \cite{cross_platform_ux}&E. Angulo amd X. Ferre perform a case study to determine the capability of cross-platform frameworks to provide a high-level user experience and usability compared to native solutions.\\
		\hline
	\end{tabular}
\end{table}
