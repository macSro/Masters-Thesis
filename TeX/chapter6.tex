% !TEX encoding = UTF-8 Unicode 
% !TEX root = praca.tex

\chapter{Implementation of sample applications}

In this chapter, the development of the sample applications is presented. The implementation is based on the research scenarios defined in chapter \ref{chap:research_scenarios}. Native solutions for Android and iOS were created using Jetpack Compose and SwiftUI, respectively.

\section{Research scenario 1: List scrolling and filtering}

The application consists of two main elements: a \emph{Switch} and a \emph{List}. Toggling the \emph{Switch} applies filtering (only every 6th element remains). The \emph{List} is generated automatically with 10000 elements and is scrollable. The optimal components (e.g. Flutter's \emph{ListView.builder}) have been chosen per each technology, therefore only visible items are rendered.

\begin{figure}[H]
    \begin{minipage}{.47\textwidth}
      \centering
      \includegraphics[height=50mm]{img/app1_kotlin}
      \caption{App 1: Kotlin (Source: Own work)}
      \label{fig:app1_kotlin}
    \end{minipage}
    \hfill
    \begin{minipage}{.47\textwidth}
      \centering
      \includegraphics[height=50mm]{img/app1_swift}
      \caption{App 1: Swift (Source: Own work)}
      \label{fig:app1_swift}
    \end{minipage}
\end{figure}

\begin{figure}[H]
  \begin{minipage}{.47\textwidth}
    \centering
    \includegraphics[height=50mm]{img/app1_flutter_android}
    \caption{App 1: Flutter Android (Source: Own work)}
    \label{fig:app1_flutter_android}
  \end{minipage}
  \hfill
  \begin{minipage}{.47\textwidth}
    \centering
    \includegraphics[height=50mm]{img/app1_flutter_ios}
    \caption{App 1: Flutter iOS (Source: Own work)}
    \label{fig:app1_flutter_ios}
  \end{minipage}
\end{figure}

\begin{figure}[H]
  \begin{minipage}{.47\textwidth}
    \centering
    \includegraphics[height=50mm]{img/app1_rn_android}
    \caption{App 1: React Native Android (Source: Own work)}
    \label{fig:app1_rn_android}
  \end{minipage}
  \hfill
  \begin{minipage}{.47\textwidth}
    \centering
    \includegraphics[height=50mm]{img/app1_rn_ios}
    \caption{App 1: React Native iOS (Source: Own work)}
    \label{fig:app1_rn_ios}
  \end{minipage}
\end{figure}

\section{Research scenario 2: Animations}

The application consists of rows of \emph{RotatingIcons} and \emph{GrowingIcons}. The former animates its rotation and the latter animates its size, with different durations and directions.

\begin{figure}[H]
  \begin{minipage}{.47\textwidth}
    \centering
    \includegraphics[height=50mm]{img/app2_kotlin}
    \caption{App 2: Kotlin (Source: Own work)}
    \label{fig:app2_kotlin}
  \end{minipage}
  \hfill
  \begin{minipage}{.47\textwidth}
    \centering
    \includegraphics[height=50mm]{img/app2_swift}
    \caption{App 2: Swift (Source: Own work)}
    \label{fig:app2_swift}
  \end{minipage}
\end{figure}

\begin{figure}[H]
\begin{minipage}{.47\textwidth}
  \centering
  \includegraphics[height=50mm]{img/app2_flutter_android}
  \caption{App 2: Flutter Android (Source: Own work)}
  \label{fig:app2_flutter_android}
\end{minipage}
\hfill
\begin{minipage}{.47\textwidth}
  \centering
  \includegraphics[height=50mm]{img/app2_flutter_ios}
  \caption{App 2: Flutter iOS (Source: Own work)}
  \label{fig:app2_flutter_ios}
\end{minipage}
\end{figure}

\begin{figure}[H]
\begin{minipage}{.47\textwidth}
  \centering
  \includegraphics[height=50mm]{img/app2_rn_android}
  \caption{App 1: React Native Android (Source: Own work)}
  \label{fig:app2_rn_android}
\end{minipage}
\hfill
\begin{minipage}{.47\textwidth}
  \centering
  \includegraphics[height=50mm]{img/app2_rn_ios}
  \caption{App 1: React Native iOS (Source: Own work)}
  \label{fig:app2_rn_ios}
\end{minipage}
\end{figure}

\section{Research scenario 3: File I/O}

The application consists of two \emph{Buttons} and a \emph{Grid View}. The first \emph{Button} saves a text file on the device and the second allows selecting images from the Gallery. Selected images are then presented in a 2-column scrollable \emph{Grid view}.

\begin{figure}[H]
  \begin{minipage}{.47\textwidth}
    \centering
    \includegraphics[height=50mm]{img/app3_kotlin}
    \caption{App 3: Kotlin (Source: Own work)}
    \label{fig:app3_kotlin}
  \end{minipage}
  \hfill
  \begin{minipage}{.47\textwidth}
    \centering
    \frame{
      \includegraphics[height=50mm]{img/app3_swift}
    }
    \caption{App 3: Swift (Source: Own work)}
    \label{fig:app3_swift}
  \end{minipage}
\end{figure}

\begin{figure}[H]
\begin{minipage}{.47\textwidth}
  \centering
  \includegraphics[height=50mm]{img/app3_flutter_android}
  \caption{App 3: Flutter Android (Source: Own work)}
  \label{fig:app3_flutter_android}
\end{minipage}
\hfill
\begin{minipage}{.47\textwidth}
  \centering
  \includegraphics[height=50mm]{img/app3_flutter_ios}
  \caption{App 3: Flutter iOS (Source: Own work)}
  \label{fig:app3_flutter_ios}
\end{minipage}
\end{figure}

\begin{figure}[H]
\begin{minipage}{.47\textwidth}
  \centering
  \includegraphics[height=50mm]{img/app3_rn_android}
  \caption{App 3: React Native Android (Source: Own work)}
  \label{fig:app3_rn_android}
\end{minipage}
\hfill
\begin{minipage}{.47\textwidth}
  \centering
  \frame{
    \includegraphics[height=50mm]{img/app3_rn_ios}
  }
  \caption{App 3: React Native iOS (Source: Own work)}
  \label{fig:app3_rn_ios}
\end{minipage}
\end{figure}

\section{Research scenario 4: Common UI elements}

The application consists of three views: \emph{Controls Page}, \emph{Form Page} and \emph{External Page}. Various common UI components are utilized as well as the bottom tab bar navigation.

\begin{figure}[H]
  \begin{minipage}{.31\textwidth}
    \centering
    \includegraphics[height=50mm]{img/app4_1_kotlin}
    \caption{App 4 (1/3): Kotlin (Source: Own work)}
    \label{fig:app4_1_kotlin}
  \end{minipage}
  \hfill
  \begin{minipage}{.31\textwidth}
    \centering
    \includegraphics[height=50mm]{img/app4_2_kotlin}
    \caption{App 4 (2/3): Kotlin (Source: Own work)}
    \label{fig:app4_2_kotlin}
  \end{minipage}
  \hfill
  \begin{minipage}{.31\textwidth}
    \centering
    \includegraphics[height=50mm]{img/app4_3_kotlin}
    \caption{App 4 (3/3): Kotlin (Source: Own work)}
    \label{fig:app4_3_kotlin}
  \end{minipage}
\end{figure}

\begin{figure}[H]
  \begin{minipage}{.31\textwidth}
    \centering
    \includegraphics[height=50mm]{img/app4_1_swift}
    \caption{App 4 (1/3): Swift (Source: Own work)}
    \label{fig:app4_1_swift}
  \end{minipage}
  \hfill
  \begin{minipage}{.31\textwidth}
    \centering
    \includegraphics[height=50mm]{img/app4_2_swift}
    \caption{App 4 (2/3): Swift (Source: Own work)}
    \label{fig:app4_2_swift}
  \end{minipage}
  \hfill
  \begin{minipage}{.31\textwidth}
    \centering
    \includegraphics[height=50mm]{img/app4_3_swift}
    \caption{App 4 (3/3): Swift (Source: Own work)}
    \label{fig:app4_3_swift}
  \end{minipage}
\end{figure}

\begin{figure}[H]
  \begin{minipage}{.31\textwidth}
    \centering
    \includegraphics[height=50mm]{img/app4_1_flutter_android}
    \caption{App 4 (1/3): Flutter Android (Source: Own work)}
    \label{fig:app4_1_flutter_android}
  \end{minipage}
  \hfill
  \begin{minipage}{.31\textwidth}
    \centering
    \includegraphics[height=50mm]{img/app4_2_flutter_android}
    \caption{App 4 (2/3): Flutter Android (Source: Own work)}
    \label{fig:app4_2_flutter_android}
  \end{minipage}
  \hfill
  \begin{minipage}{.31\textwidth}
    \centering
    \includegraphics[height=50mm]{img/app4_3_flutter_android}
    \caption{App 4 (3/3): Flutter Android (Source: Own work)}
    \label{fig:app4_3_flutter_android}
  \end{minipage}
\end{figure}

\begin{figure}[H]
  \begin{minipage}{.31\textwidth}
    \centering
    \includegraphics[height=50mm]{img/app4_1_flutter_ios}
    \caption{App 4 (1/3): Flutter iOS (Source: Own work)}
    \label{fig:app4_1_flutter_ios}
  \end{minipage}
  \hfill
  \begin{minipage}{.31\textwidth}
    \centering
    \includegraphics[height=50mm]{img/app4_2_flutter_ios}
    \caption{App 4 (2/3): Flutter iOS (Source: Own work)}
    \label{fig:app4_2_flutter_ios}
  \end{minipage}
  \hfill
  \begin{minipage}{.31\textwidth}
    \centering
    \includegraphics[height=50mm]{img/app4_3_flutter_ios}
    \caption{App 4 (3/3): Flutter iOS (Source: Own work)}
    \label{fig:app4_3_flutter_ios}
  \end{minipage}
\end{figure}

\begin{figure}[H]
  \begin{minipage}{.31\textwidth}
    \centering
    \includegraphics[height=50mm]{img/app4_1_rn_android}
    \caption{App 4 (1/3): React Native Android (Source: Own work)}
    \label{fig:app4_1_rn_android}
  \end{minipage}
  \hfill
  \begin{minipage}{.31\textwidth}
    \centering
    \includegraphics[height=50mm]{img/app4_2_rn_android}
    \caption{App 4 (2/3): React Native Android (Source: Own work)}
    \label{fig:app4_2_rn_android}
  \end{minipage}
  \hfill
  \begin{minipage}{.31\textwidth}
    \centering
    \includegraphics[height=50mm]{img/app4_3_rn_android}
    \caption{App 4 (3/3): React Native Android (Source: Own work)}
    \label{fig:app4_3_rn_android}
  \end{minipage}
\end{figure}

\begin{figure}[H]
  \begin{minipage}{.31\textwidth}
    \centering
    \includegraphics[height=50mm]{img/app4_1_rn_ios}
    \caption{App 4 (1/3): React Native iOS (Source: Own work)}
    \label{fig:app4_1_rn_ios}
  \end{minipage}
  \hfill
  \begin{minipage}{.31\textwidth}
    \centering
    \includegraphics[height=50mm]{img/app4_2_rn_ios}
    \caption{App 4 (2/3): React Native iOS (Source: Own work)}
    \label{fig:app4_2_rn_ios}
  \end{minipage}
  \hfill
  \begin{minipage}{.31\textwidth}
    \centering
    \includegraphics[height=50mm]{img/app4_3_rn_ios}
    \caption{App 4 (3/3): React Native iOS (Source: Own work)}
    \label{fig:app4_3_rn_ios}
  \end{minipage}
\end{figure}

\section{Key takeaways obtained during the implementation process}

As considered in Chapter \ref{chap:cross_mobile_dev}, one of the significant aspects of cross-platform framework development is the ability to acquire the native look and feel of mobile applications. Of course, this does not concern native approaches. As can be observed in the screenshots above, mobile apps implemented with Flutter tend to be almost completely identical to their native equivalents. In the case of React Native, iOS apps seem to be quite similar, although not as much as in the case of Flutter. However, a big difference can be noticed when comparing Android apps. The reason for that is that \emph{Material Design 3} is currently the primary design system for user interfaces, and React Native has not yet adjusted to it. Moreover, the process of theming is easier in Flutter because it is set up for both platforms by default, which is not true for React Native. On the other hand, Flutter requires more code in order to accomplish this native look and feel. For example, \mintinline{dart}{TextField} and \mintinline{dart}{CupertinoTextField} widgets should be used for Android and iOS, respectively, for this purpose. This is usually achieved by creating custom widgets, e.g., \mintinline{dart}{MyCustomTextField} which renders the correct widget according to the running operating system and removes the necessity to check the platform each time a text field is used across the whole application.

During the implementation, the official documentation as well as the community posts could be used as the main sources of knowledge. The development process carried out for the purpose of this thesis provided information for the comparison regarding this context. Both SwiftUI and Flutter offer the most extensible documentation, which enables quick understanding of new concepts. Next in line is React Native which provides a less detailed description of some aspects but is still not lacking. Finally, Jetpack Compose documentation comes across as the most general, which can result in the necessity of further research in order to gather enough knowledge for the implementation of a specific component.

\clearpage
